\documentclass[12pt]{memoir}

\usepackage[letterpaper,left=3.0cm,right=2.5cm,top=3.0cm,bottom=3.0cm]{geometry}

\usepackage{lmodern}

\usepackage{amsmath}
\usepackage{slashed}
% \usepackage{fontspec}
% \setmainfont{Latin Modern}
% \setsansfont{Source Sans Pro}
% \setmathrm{Latin Modern}
% \setmonofont{Consolas}

% Remove this package
% \usepackage{lipsum}

% acronyms
% \usepackage{acronym}

\usepackage{hyperref}

% For license
\usepackage[
    type={CC},
    modifier={by},
    version={4.0},
]{doclicense}

\setsecnumdepth{subsubsection}
\maxtocdepth{subsection}

% Parskip-style
\setlength{\parindent}{0pt}
\nonzeroparskip

\begin{document}

\include{macros}

% \title{A sample set of notes}

% \author{Matthias Heinz}
\frontmatter
% \maketitle
% \titlecustom{A sample set of notes}{Some subtitle}{Matthias Heinz}

\begin{titlingpage}

\newcommand{\HRule}{\rule{\linewidth}{0.5mm}} % Defines a new command for the horizontal lines, change thickness here

\center % Center everything on the page

%----------------------------------------------------------------------------------------
%	HEADING SECTIONS
%----------------------------------------------------------------------------------------

\textsc{\LARGE TU Darmstadt}\\[1.5cm] % Name of your university/college
% \includegraphics[scale=.1]{Uppsala_University_seal_svg.png}\\[1cm] % Include a department/university logo - this will require the graphicx package
% \textsc{\Large Course name}\\[0.5cm] % Major heading such as course name
% \textsc{\large Course code}\\[0.5cm] % Minor heading such as course title

%----------------------------------------------------------------------------------------
%	TITLE SECTION
%----------------------------------------------------------------------------------------

\HRule \\[0.4cm]
{ \huge \bfseries Advanced topics in quantum field theory}\\[0.4cm] % Title of your document
\HRule \\[1.5cm]

%----------------------------------------------------------------------------------------
%	AUTHOR SECTION
%----------------------------------------------------------------------------------------

\begin{minipage}{0.4\textwidth}
\begin{flushleft} \large
\emph{Author:}\\
Matthias \textsc{Heinz}\\ % Your name
\end{flushleft}

\end{minipage}\\[2cm]

% If you don't want a supervisor, uncomment the two lines below and remove the section above
%\Large \emph{Author:}\\
%John \textsc{Smith}\\[3cm] % Your name

%----------------------------------------------------------------------------------------
%	DATE SECTION
%----------------------------------------------------------------------------------------

{\large \today}\\[2cm] % Date, change the \today to a set date if you want to be precise

\vfill % Fill the rest of the page with whitespace

\end{titlingpage}

% Public abstract

% abstract
\chapter{Abstract}
These notes grew out of the Quantum Field Theory 2 course
taught by Prof.~Bubablla at the TU Darmstadt.
By writing these notes, I hope to better understand the material.

% License
\chapter{License}
\doclicenseThis{}

% Dedication

% Acknowledgements

% Preface

\cleardoublepage\tableofcontents

\cleardoublepage\listoftables

\cleardoublepage\listoffigures

% Uncomment for acronyms
% use acronym in text with \ac{label} or \acl{label}
% \cleardoublepage\chapter{List of Acronyms}
\begin{acronym}
  % Declare acronyms like:
  % \acro{nsf}[NSF]{National Science Foundation}
\end{acronym}


% Can add list of symbols here
% \cleardoublepage\include{notes/symbols}

\mainmatter{}

\chapter{Renormalization}

\section{Key ideas}

\begin{itemize}
  \item Calculation of $n$-point functions beyond tree-level requires the evaluation of Feynman diagrams
    containing loops, which involves the evaluation of divergent integrals over free intermediate momenta
  \item Underlying this is a misidentification of parameters in the Lagrangian with observed values,
    which is correct at tree-level but not in general
  \item Since these ``bare'' parameters are free, we can choose them to be such that
    the measured values of certain observables are correct,
    such as the mass and/or the coupling (we will see that this must be fixed to a specific momentum scale)
  \item When considering amputated scattering processes, there are two kinds of corrections:
    vacuum polarization corrections to propagators and vertex corrections
  \item The perturbative evaluation of these corrections
    and the fixing of Lagrangian parameters to give correct results
    based on certain matching conditions is \textit{renormalization}
\end{itemize}

\section{Dressing the vertex and the propagator}

\section{Counter-term renormalization}

As renormalization is concerned with managing divergences,
it is useful to have a way to estimate the degree of divergence of diagrams.
The superficial degree of divergence $D$ allows us to do this,
relating divergence to loops (integrals over momenta) and propagators.
This can be simplified to an expression containing only the number of external lines.

Additionally, we can focus on one-point irreducible (1PI) diagrams,
which ultimately are what contributes to the divergence of larger diagrams containing these 1PI diagrams.
These two realizations come together to constrain the number of divergences we need to consider
to only a few 1PI diagrams.

Some of these can be ruled out due to physical or symmetry reasons.
For example, vacuum bubbles are always divergent, but can be ignore because they cancel in correlation functions.
1PI diagrams with only odd numbers of photon lines are 0 by charge conjugation.
By the Ward identity, the 4-photon 1PI diagram must give 0 when a photon line is replaced by its momentum.
This ultimately means that when expanding in momenta the first term is 0 and the non-vanishing term has $D=4$.
Thus, this diagram is not actually divergent.

For QED, this means we have 3 relevant divergence 1PI amplitudes,
related to the electron self-energy, electron-photon vertex, and photon polarization.

\chapter{The renormalization group}

\section{Key ideas}

\begin{itemize}
  \item Previously, we saw that certain renormalization conditions required a specific kinematics to be specified
    (for example, 4-momentum transfer squared $q^2=0$)
  \item This means that our calculation makes non-trivial predictions about the behavior of the theory away from these points
  \item A first naive approach involves treating the momentum dependence of the coupling,
    of QED for example,
    away from our renormalization point as an effective coupling
  \item A similar result can be arrived at by doing a sophisticated analysis of scaling in the theory
  \item Some general results that can be obtained by this analysis are the $\beta$ function for the coupling
    and the $\gamma$ functions, which predict anomalous scaling behavior in the theory
\end{itemize}

\section{The running coupling of QED}

\section{The Callan-Symanzik equation}

\section{The beta function}

\chapter{The path integral}

\section{Key ideas}

\begin{itemize}
  \item So far, our approach to quantization and evaluation of observables has been
    to quantize the theory by demanding canonical (anti-) commutation relations
    and compute observables perturbatively by evaluating the relevant diagrams by hand
  \item An alternative approach to quantization occurs via the path integral,
    also applicable to classical systems
  \item This formalism contains all correlation functions in one object,
    allowing the calculation of observables in principle up to complete order
  \item The central object is the generating functional,
    which couples classical sources to fields of the theory
  \item Some difficulties arise when dealing with the quantization of fermions,
    which require the introduction of Grassman fields to ensure anti-commuting behavior
  \item The path integral also allows for the easy identification of the propagator,
    but for massless gauge fields this is spoiled by the fact that the inverse propagtor is singular
  \item Fixing a gauge makes the inverse propagator invertible (recall previously we worked in the Feynman gauge)
  \item More generally, the Faddeev-Popov method allows for the fixing of a gauge
    for the functional integral over gauge fields,
    which results in the modification of the inverse propagator by a gauge-fixing term
  \item Specific choices for this gauge-fixing-term parameter $\xi$ correspond to different gauges,
    and we recover the Feynman gauge result used previously
\end{itemize}

\section{Application to scalar field theory}

\subsection{Classical case}

\subsection{Path integral quantization}

\subsection{Generating functional}

\section{Application to QED}

\subsection{Quantizing fermions}

\subsection{Quantizing the electromagnetic gauge field}

\section{Symmetries and Ward identities}

\section{Extending the generating functional concept}

\subsection{The Schwinger functional}

\subsection{The effective action}

\subsection{Useful properties}

\chapter{Spontaneous symmetry breaking}

\section{Key ideas}

\begin{itemize}
  \item Spontaneous symmetry breaking (SSB) is when a theory has a global symmetry
    that is not realized in the ground state
  \item A simple classical example is the sombrero potential,
    which has a 2-dimensional rotational symmetry that the degenerate ground states break
  \item This has the consequence that for the broken symmetries
    there are 0-energy excitations from the ground state into another ground state,
    which in a quantum field theory view would correspond to massless particles
  \item One such theory is the linear $\sigma$ model,
    in which the theory has an $\text{O}(N)$ symmetry,
    but the ground state only has an $\text{O}(N-1)$ symmetry,
    giving rise to $N-1$ ``Goldstone bosons''
  \item Quantizing the theory and renormalizing it gives the result that
    these bosons are still massless and the Goldstone theorem remains intact
\end{itemize}

\section{Classical argument}

The Lagrangian for the linear $\sigma$ model in Eq.~\ref{eq:linear_sigma_model} has a couple of interesting/unusual properties:
\begin{enumerate}
  \item the ``mass term'' has the wrong sign,
  \item and the Lagrangian is invariant underr $N$-dimensional rotations ($\text{O}(N)$).
\end{enumerate}
Using $\mathcal{H} = \pi^2 - \mathcal{L}$, we can identify the classical potential with
\begin{equation}
  V(\Phi^2) = - \frac{1}{2} \mu^2 \Phi^2 + \frac{\lambda}{4} \Phi^4\,.
\end{equation}
The classical ground state is a constant field $\lvert\Phi\rvert = \nu$ that minimizes the potential,
which happens for
\begin{equation}
  \nu^2 = \frac{\mu^2}{\lambda}\,.
\end{equation}
It is easy to see that this ground state is not unique and not invariant under an $N$-dimensional rotation.

The consequences of this are made clear when we shift our fields to work around one of our classical minima:
\begin{align}
  \Phi^i(x) & = \pi^i(x)\,,  &i \in \left\{1,\ldots,N-1\right\}\\
  \Phi^N(x) &=  \nu + \sigma(x)\,. &
\end{align}
Reexpressing the Lagrangian in terms of these shifted fields shows that only the $\sigma$ mode has a ``mass'' term,
and the $N-1$ $\pi$ fields correspond to 0-energy excitations of the ground state.

This is one realization of the Goldstone theorem:
For every spontaneously broken continuous symmetry there exists a 0-energy excited state.
In a quantized theory, we would identify these with massless particles, the so-called ``Goldstone bosons.''
The number of unique 0-energy excitations can be determined via the reduction of generators
going from the full Lagrangian symmetry group to the symmetry group of the ground state.
In this example, the group $\text{O}(N)$ has $\frac{1}{2}(N-1)N$ generators,
so this would mean we get $N-1$ 0-energy excitations, which is what we just observed.

The classical proof of this can be obtained schematically as follows:
\begin{enumerate}
  \item Expanding the potential around a minimum $\Phi_0^i$, we identify the 2-derivative matrix as a mass matrix $m^2_{ab}$.
  \item If we perform an infinitesimal symmetry transformation of the fields
    (which leaves the potential unchanged)
    and evaluate at the minimum,
    we find ultimately that the product of the mass matrix and the vector giving the shift of the fields must be 0.
  \item In the case that the rotation is around an axis not equal to $\Phi_0^i$, this is a non-trivial condtion,
    which means that the infinitesimal shift vector is an eigenvector of the mass matrix with eigenvalue 0.
\end{enumerate}



\section{Quantization and renormalization}

\chapter{The gauge principle}

\section{Key ideas}

\begin{itemize}
  \item Previously, we arrived at the QED Lagrangian by explicitly coupling a field to our fermions,
    adding a kinetic term for the new field,
    and demanding specific transformation behavior under gauge transformations
  \item An alternative approach is to take the free Lagrangian with a global symmetry,
    $\text{U}(1)$ in the case of QED,
    promote this to a local symmetry and require that the Lagrangian is invariant under it
  \item This approach, called ``gauging,'' naturally generates the gauge fields and ultimately gives us the same result,
    but it gives a way to systematically construct the QED Lagrangian as well as more complicated theories
\end{itemize}

\section{QED from gauge invariance}

Starting from the free Dirac Lagrangian in Eq.~\ref{eq:dirac_free},
we want to get the QED Lagrangian
that is invariant under a local $\text{U}(1)$ gauge transformation of the fields
\begin{equation}
  \Psi(x) \rightarrow e^{i \alpha(x)}\Psi(x).
\end{equation}

The approach taken is as follows:
\begin{enumerate}
  \item The mass term $\bar{\Psi}m\Psi$ is already invariant.
    The derivative as written in an $\epsilon$ limit is not,
    since any two spacetime points may transform differently.
  \item Thus we define an object $U(y, x)$
    which, when transformed,
    undoes the transformation at one spacetime point $x$
    and applies the transformation at another spacetime point $y$.
    Applying this to our $\epsilon$-expanded derivative, we define a covariant derivative,
    which transforms consistently.
  \item Expanding everything in powers of $\epsilon$ give a normal derivative term
    and a non-derivative term proportional to the derivative of $U$ at $x$.
    This can be defined into a new field whose transformation behavior is given by the transformation behavior of $U$.
  \item This gives us covariant derivative of the field, which transforms just like the field itself.
    Now one can combine fields and derivatives of fields to produce gauge invariant building blocks.
  \item The final step involves obtaining the kinetic term for the new field.
    One possible way to arrive at this is noting that $[D_{\mu}, D_{\nu}]$ is invariant under gauge transformations.
    This term can be simplified into $F_{\mu\nu}=\partial_{\mu}A_{\nu} - \partial_{\nu}A_{\mu}$.
    Lorentz invariance requires that this only appears as $(F_{\mu\nu})^2$ or contracted with a Levi-Civita tensor,
    and the latter term can be shown not to contribute to the action and can be left out as a result.
\end{enumerate}

\section{The Higgs mechanism}

\chapter{Non-abelian gauge theories}

\section{Key ideas}

\begin{itemize}
  \item When gauging QED, our final expressions were simplified by the fact that the generators of $\text{U}(1)$ commute
  \item In non-abelian gauge theories, the symmetry group is non-abelian,
    meaning the group elements (and thus the generators of the group) do not in general commute
  \item This leads to some additional complexities when gauging these theories
\end{itemize}

\section{Yang-Mills theory}

\section{QCD}

The ``free'' QCD Lagrangian is given by
\begin{equation}\label{eq:qcd_free}
  \mathcal{L} = \sum_{f} \bar{\Psi}_f (i \slashed{\partial} - m_f) \Psi_f\,,
\end{equation}
where our $\Psi$ fields are triplets of Dirac spinors, corresponding to color degrees of freedom.

\subsection{Symmetries of QCD}

The Lagrangian above has a global $\text{SU}(3)$ symmetry,
which we will promote to a local gauge symmetry.
The $\text{SU}(3)$ generators are $T^a=\lambda^a / 2$, which means we now demand invariance under
\begin{equation}
  \Psi_f(x) \rightarrow \exp(i\theta^a(x)T^a)\Psi_f(x).
\end{equation}
This gives the covariant derivative
\begin{equation}
  D_{\mu} = \partial_{\mu} - i g A_{\mu}^a(x) T^a,
\end{equation}
where we have generated the 8 gluon gauge fields.
The transformation behavior of these fields is a bit more complicated,
\begin{equation}
  A_{\mu}^a(x) \rightarrow A_{\mu}^a(x) + \frac{1}{g} \partial_{\mu} \theta^a(x) + f^{abc}A_{\mu}^b(x)\theta^c(x),
\end{equation}
directly a result of the fact that the generators do not commute.

Once again, as in QED, the field strength tensor
\begin{equation}
  G_{\mu\nu}^a = \partial_{\mu} A_{\nu}^a - \partial_{\nu} A_{\mu}^a  + g f^{abc} A_{\mu}^b A_{\nu}^c
\end{equation}
is invariant under local transformations.
Thus the Lorentz invariant term we can include in our gauged QCD Lagrangian is $G_{\mu\nu}^a G^{\mu\nu a}$.
The first 2 terms in the field strength tensor gives us a kinetic term for the gluons.
However, the final term in $G_{\mu\nu}$ generates 3-gluon ($\mathcal{O}(g)$) and 4-gluon terms ($\mathcal{O}(g^2)$), due to its quadratic appearance.


\subsection{Quantization}


\begin{appendices}
  \chapter{Some useful equations to remember}

  \section{Lagrangians}

  Free Klein-Gordon theory:
  \begin{equation}\label{eq:kg_free}
    \mathcal{L} = \frac{1}{2}(\partial^{\mu} \Phi)(\partial_{\mu} \Phi) - \frac{m^2}{2} \Phi^2\,.
  \end{equation}

  $\Phi^4$ theory:
  \begin{equation}\label{eq:phi4}
    \mathcal{L} = \frac{1}{2}(\partial^{\mu} \Phi)(\partial_{\mu} \Phi) - \frac{m^2}{2} \Phi^2 - \frac{\lambda}{4!} \Phi^4\,.
  \end{equation}

  Linear $\sigma$ model:
  \begin{equation}\label{eq:linear_sigma_model}
    \mathcal{L} = \frac{1}{2}(\partial^{\mu} \Phi^i)(\partial_{\mu} \Phi^i) + \frac{\mu^2}{2} \Phi^2 - \frac{\lambda}{4} \Phi^4\,,
  \end{equation}
  with $\Phi^i$ an $N$-dimensional vector of scalar fields.

  Free Dirac theory:
  \begin{equation}\label{eq:dirac_free}
    \mathcal{L} = \bar{\Psi}(i\slashed{\partial} - m)\Psi\,.
  \end{equation}

  QED:
  \begin{equation}\label{eq:qed}
    \mathcal{L} = \bar{\Psi}(i\slashed{D} - m)\Psi - \frac{1}{4}(F_{\mu\nu})^2 \,,
  \end{equation}
  with
  \begin{equation}\label{eq:qed_cov_deriv}
    D_{\mu} = \partial_{\mu} + i e A_{\mu}\,.
  \end{equation}

  \chapter{The spectral representation}

\end{appendices}

\backmatter{}

% \bibliographystyle{alpha} % use your favorite BIBTeX style
% \nocite{*} % To display all refs, even uncited refs (useful when editting)
% \bibliography{notes/notesbib}

\end{document}
