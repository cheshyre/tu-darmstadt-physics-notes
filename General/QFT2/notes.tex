\documentclass[12pt]{memoir}

\usepackage[letterpaper,left=3.0cm,right=2.5cm,top=3.0cm,bottom=3.0cm]{geometry}

\usepackage{lmodern}
% \usepackage{fontspec}
% \setmainfont{Latin Modern}
% \setsansfont{Source Sans Pro}
% \setmathrm{Latin Modern}
% \setmonofont{Consolas}

% Remove this package
% \usepackage{lipsum}

% acronyms
% \usepackage{acronym}

\usepackage{hyperref}

% For license
\usepackage[
    type={CC},
    modifier={by},
    version={4.0},
]{doclicense}

\setsecnumdepth{subsubsection}
\maxtocdepth{subsection}

\begin{document}

\include{macros}

% \title{A sample set of notes}

% \author{Matthias Heinz}
\frontmatter
% \maketitle
% \titlecustom{A sample set of notes}{Some subtitle}{Matthias Heinz}

\begin{titlingpage}

\newcommand{\HRule}{\rule{\linewidth}{0.5mm}} % Defines a new command for the horizontal lines, change thickness here

\center % Center everything on the page

%----------------------------------------------------------------------------------------
%	HEADING SECTIONS
%----------------------------------------------------------------------------------------

\textsc{\LARGE TU Darmstadt}\\[1.5cm] % Name of your university/college
% \includegraphics[scale=.1]{Uppsala_University_seal_svg.png}\\[1cm] % Include a department/university logo - this will require the graphicx package
% \textsc{\Large Course name}\\[0.5cm] % Major heading such as course name
% \textsc{\large Course code}\\[0.5cm] % Minor heading such as course title

%----------------------------------------------------------------------------------------
%	TITLE SECTION
%----------------------------------------------------------------------------------------

\HRule \\[0.4cm]
{ \huge \bfseries Advanced topics in quantum field theory}\\[0.4cm] % Title of your document
\HRule \\[1.5cm]

%----------------------------------------------------------------------------------------
%	AUTHOR SECTION
%----------------------------------------------------------------------------------------

\begin{minipage}{0.4\textwidth}
\begin{flushleft} \large
\emph{Author:}\\
Matthias \textsc{Heinz}\\ % Your name
\end{flushleft}

\end{minipage}\\[2cm]

% If you don't want a supervisor, uncomment the two lines below and remove the section above
%\Large \emph{Author:}\\
%John \textsc{Smith}\\[3cm] % Your name

%----------------------------------------------------------------------------------------
%	DATE SECTION
%----------------------------------------------------------------------------------------

{\large \today}\\[2cm] % Date, change the \today to a set date if you want to be precise

\vfill % Fill the rest of the page with whitespace

\end{titlingpage}

% Public abstract

% abstract
\chapter{Abstract}
These notes grew out of the Quantum Field Theory 2 course
taught by Prof.~Bubablla at the TU Darmstadt.
By writing these notes, I hope to better understand the material.

% License
\chapter{License}
\doclicenseThis{}

% Dedication

% Acknowledgements

% Preface

\cleardoublepage\tableofcontents

\cleardoublepage\listoftables

\cleardoublepage\listoffigures

% Uncomment for acronyms
% use acronym in text with \ac{label} or \acl{label}
% \cleardoublepage\chapter{List of Acronyms}
\begin{acronym}
  % Declare acronyms like:
  % \acro{nsf}[NSF]{National Science Foundation}
\end{acronym}


% Can add list of symbols here
% \cleardoublepage\include{notes/symbols}

\mainmatter{}

\chapter{Renormalization}

\chapter{The renormalization group}

\chapter{The path integral}

\chapter{Spontaneous symmetry breaking}

\chapter{The gauge principle}

\chapter{Non-abelian gauge theories}


\begin{appendices}
  \chapter{The spectral representation}

\end{appendices}

\backmatter{}

% \bibliographystyle{alpha} % use your favorite BIBTeX style
% \nocite{*} % To display all refs, even uncited refs (useful when editting)
% \bibliography{notes/notesbib}

\end{document}
