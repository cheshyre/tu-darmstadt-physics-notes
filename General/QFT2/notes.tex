\documentclass[12pt]{memoir}

\usepackage[letterpaper,left=3.0cm,right=2.5cm,top=3.0cm,bottom=3.0cm]{geometry}

\usepackage{lmodern}

\usepackage{amsmath}
\usepackage{slashed}
% \usepackage{fontspec}
% \setmainfont{Latin Modern}
% \setsansfont{Source Sans Pro}
% \setmathrm{Latin Modern}
% \setmonofont{Consolas}

% Remove this package
% \usepackage{lipsum}

% acronyms
% \usepackage{acronym}

\usepackage{hyperref}

% For license
\usepackage[
    type={CC},
    modifier={by},
    version={4.0},
]{doclicense}

\setsecnumdepth{subsubsection}
\maxtocdepth{subsection}

% Parskip-style
\setlength{\parindent}{0pt}
\nonzeroparskip

\begin{document}

\include{macros}

% \title{A sample set of notes}

% \author{Matthias Heinz}
\frontmatter
% \maketitle
% \titlecustom{A sample set of notes}{Some subtitle}{Matthias Heinz}

\begin{titlingpage}

\newcommand{\HRule}{\rule{\linewidth}{0.5mm}} % Defines a new command for the horizontal lines, change thickness here

\center % Center everything on the page

%----------------------------------------------------------------------------------------
%	HEADING SECTIONS
%----------------------------------------------------------------------------------------

\textsc{\LARGE TU Darmstadt}\\[1.5cm] % Name of your university/college
% \includegraphics[scale=.1]{Uppsala_University_seal_svg.png}\\[1cm] % Include a department/university logo - this will require the graphicx package
% \textsc{\Large Course name}\\[0.5cm] % Major heading such as course name
% \textsc{\large Course code}\\[0.5cm] % Minor heading such as course title

%----------------------------------------------------------------------------------------
%	TITLE SECTION
%----------------------------------------------------------------------------------------

\HRule \\[0.4cm]
{ \huge \bfseries Advanced topics in quantum field theory}\\[0.4cm] % Title of your document
\HRule \\[1.5cm]

%----------------------------------------------------------------------------------------
%	AUTHOR SECTION
%----------------------------------------------------------------------------------------

\begin{minipage}{0.4\textwidth}
\begin{flushleft} \large
\emph{Author:}\\
Matthias \textsc{Heinz}\\ % Your name
\end{flushleft}

\end{minipage}\\[2cm]

% If you don't want a supervisor, uncomment the two lines below and remove the section above
%\Large \emph{Author:}\\
%John \textsc{Smith}\\[3cm] % Your name

%----------------------------------------------------------------------------------------
%	DATE SECTION
%----------------------------------------------------------------------------------------

{\large \today}\\[2cm] % Date, change the \today to a set date if you want to be precise

\vfill % Fill the rest of the page with whitespace

\end{titlingpage}

% Public abstract

% abstract
\chapter{Abstract}
After an introductory course in quantum field theory
that introduces canonical quantization,
tree-level Feynman diagrams and rules,
and the procedures for calculating processes with these,
it is time to look at advanced topics in quantum field theories.
In these notes, we consider:
going beyond tree-level in calculations and the general need for renormalization;
path integral quantization of quantum field theories and various associated methods;
the modern approach of ``gauging'' a free theory to get an interacting one;
and, some preliminary discussions regarding non-abelian gauge theories.

These notes grew out of the Quantum Field Theory 2 course
taught by Prof.~Bubablla at the TU Darmstadt.
By writing these notes, I hope to better understand the material.

% License
\chapter{License}
\doclicenseThis{}

% Dedication

% Acknowledgements

% Preface

\cleardoublepage\tableofcontents

% \cleardoublepage\listoftables

% \cleardoublepage\listoffigures

% Uncomment for acronyms
% use acronym in text with \ac{label} or \acl{label}
% \cleardoublepage\chapter{List of Acronyms}
\begin{acronym}
  % Declare acronyms like:
  % \acro{nsf}[NSF]{National Science Foundation}
\end{acronym}


% Can add list of symbols here
% \cleardoublepage\include{notes/symbols}

\mainmatter{}

\chapter{Renormalization}

\section{Key ideas}

\begin{itemize}
  \item Calculation of $n$-point functions beyond tree-level requires the evaluation of Feynman diagrams
    containing loops, which involves the evaluation of divergent integrals over free intermediate momenta
  \item Underlying this is a misidentification of parameters in the Lagrangian with observed values,
    which is correct at tree-level but not in general
  \item Since these ``bare'' parameters are free, we can choose them to be such that
    the measured values of certain observables are correct,
    such as the mass and/or the coupling (we will see that this must be fixed to a specific momentum scale)
  \item When considering amputated scattering processes, there are two kinds of corrections:
    vacuum polarization corrections to propagators and vertex corrections
  \item The perturbative evaluation of these corrections
    and the fixing of Lagrangian parameters to give correct results
    based on certain matching conditions is \textit{renormalization}
\end{itemize}

\section{Dressing the vertex and the propagator}

\section{Counter-term renormalization}

As renormalization is concerned with managing divergences,
it is useful to have a way to estimate the degree of divergence of diagrams.
The superficial degree of divergence $D$ allows us to do this,
relating divergence to loops (integrals over momenta) and propagators.
This can be simplified to an expression containing only the number of external lines.

Additionally, we can focus on one-point irreducible (1PI) diagrams,
which ultimately are what contributes to the divergence of larger diagrams containing these 1PI diagrams.
These two realizations come together to constrain the number of divergences we need to consider
to only a few 1PI diagrams.

Some of these can be ruled out due to physical or symmetry reasons.
For example, vacuum bubbles are always divergent, but can be ignored because they cancel in correlation functions.

We will now focus on QED, where we have 6 divergent 1PI amplitudes and the vacuum bubbles.
1PI diagrams with only odd numbers of photon lines are 0 by charge conjugation.
By the Ward identity, the 4-photon 1PI diagram must give 0 when a photon line is replaced by its momentum.
This ultimately means that when expanding in momenta the first term is 0 and the non-vanishing term has $D=-4$.
Thus, this diagram is not actually divergent.

For QED, this means we have 3 relevant divergent 1PI amplitudes,
related to the electron self-energy, electron-photon vertex, and photon polarization.

Consider the electron self-energy.
The first naive contribution to this is a fermion line with a photon going off and returning to the line.
Naively this amplitude should be linearly divergent ($\sim d^4k (1/k^2) (1/k)$).
Writing a Taylor series for the amplitude yields the result that
we have a linearly and a logarithmically divergent term and then a finite term.
In fact, chiral symmetry requires that the linearly divergent term is actually only logarithmically divergent
($\sim m \log(\Lambda)$).
So we get two logarithmically divergent coefficients (one from $p \log(\Lambda)$, one from $m \log(\Lambda)$)
from the electron self-energy.

The electron-photon vertex requires two electron propagators and one photon propagator at the first loop order.
Thus it is naively (and also actually) logarithmically divergent.
The photon polarization has an electron loop at NLO, giving a naive estimate of quadratic divergence.
Once again, the Ward identity constrains the form of the photon polarization,
and a Taylor series shows that the logarithmically divergent term is the only divergent term that does not vanish.

As a result we have 3 divergent 1PI amplitudes with 4 unique divergent terms.
In our theory, we have 4 undetermined parameters,
the field strength renormalization coefficients for the fermion and gauge fields,
the fermion mass,
and the coupling strength (or the fermion charge).
So we have everything we need to manage the existing divergences and can proceed to do counter term renormalization.

Our first goal is to re-express the Lagrangian initially given in terms of bare parameters
using renormalized quantities.
The field strength renormalization appears in the numerator of the propagator ($\sim Z/k^2$),
which is a 2-point function,
so one can absorb this by defining renormalized fields with a square root of this factor
relative to the bare fields.
We then define the renormalized mass and charge in terms of field strength renormalization coefficients
and an additional renormalization coefficient to give a simple form of the Lagrangian.
Then we take each coefficient and write it as $1 + \delta$,
where the $1$ gives the original Lagrangian term but with renormalized quantities
and the $\delta$ gives the corresponding counter term.

These counter terms add Feynman rules to the theory.
First, every counter term has terms of all orders (besides $\alpha^0$) in our perturbative expansion.
Thus, at NLO we get regular loop diagrams and first order counter terms,
and at N2LO we get regular second order loop diagrams,
first order loop diagrams with a first order counter term,
and second order counter terms.

These Feynman rules can be used to perturbatively calculate certain quantities.
Which quantities we should calculate first is fixed by our renormalization conditions,
which specify how we fix the counter term coefficients based on certain observables.
For QED, one option for renormalization conditions is:
\begin{enumerate}
  \item We require the electron mass from the dressed propagator to be the physical electron mass.
    This means we want the pole of the propagator to be at $p=m$,
    which constrains the constant term in the Taylor series of the electron self-energy $\Sigma$.
  \item We require the residue of the electron propagator to be 1.
    This constrains the derivative of $\Sigma$.
  \item We require the residue of the photon propagator to be 1.
    Note: these residue constraints correspond to particle number constraints.
  \item We require the electron-photon vertex at zero momentum transfer to be given by the expected formula
    $e \gamma^{\mu}$.
\end{enumerate}

A couple more remarks to some subtleties in the remaining calculation that won't be worked out here.
The photon propagator is infrared (IR) divergent,
meaning that parts of integrals over small momenta lead to divergences too.
An approach to regularizing this is giving the photon a mass $\mu$.
This regularization won't go away, and the only option is to directly calculate observables
and see that results are independent of the regularizing mass.

During the calculation, the integral for the electron self-energy eventually becomes
\begin{equation}
  \int d^4l_{\text{E}} \frac{1}{(l_{\text{E}}^2 + \Delta(x) - i \epsilon)^2}\,,
\end{equation}
where $\Delta$ depends on our Feynman parameter, the masses, and the momentum of the propagator
and can become negative.
When this happens, the self-energy will get an imaginary part,
which corresponds exactly to the condition that the momentum $p$ is enough to produce an on-shell electron with mass $m$ and an on-shell photon with mass $\mu$.
This is a general result and is discussed more in the spectral representation of the propagator.

Finally, the regularization procedure is also something to consider carefully.
In the case of these calculations, a simple momentum cutoff is not okay, because it would disrupt some symmetries.
What this means concretely is that Ward identities,
which can be derived based on symmetries alone,
may no longer be true after improper regularization.
The approach taken by Pauli and Villars is to subtract from the term above another term,
replacing the mass $m$ with a cutoff $\Lambda$.
This means the second term has the same high-momentum behavior, so it cancels the UV divergence,
because $\Delta$ is
\begin{equation}
  \Delta = m^2 x + \mu^2 (1 - x) - p^2 x (1 - x)\,.
\end{equation}
Eventually, one can take $\Lambda \rightarrow \infty$, and this added term won't contribute.

The evaluation of these NLO integrals is beyond these notes,
but they are done in gory detail in Ref.~\cite{Peskin:1995ev}.

\chapter{The renormalization group}

\section{Key ideas}

\begin{itemize}
  \item Previously, we saw that certain renormalization conditions required a specific kinematics to be specified
    (for example, 4-momentum transfer squared $q^2=0$)
  \item This means that our calculation makes non-trivial predictions about the behavior of the theory away from these points
  \item A first naive approach involves treating the momentum dependence of the coupling,
    of QED for example,
    away from our renormalization point as an effective coupling
  \item A similar result can be arrived at by doing a sophisticated analysis of scaling in the theory
  \item Some general results that can be obtained by this analysis are the $\beta$ function for the coupling
    and the $\gamma$ function,
    which predict scale dependence and anomalous scaling behavior in the theory
\end{itemize}

\section{The running coupling of QED}

A concrete evaluation of the dressed electron-photon vertex and the photon polarization
with the renormalization conditions discussed gave
\begin{equation}
  e = (1 + \Pi_2(q^2=0))^{1/2}e_0\,,
\end{equation}
where $\Pi_2$ is the photon polarization (without the $g^{\mu\nu}q^2 - q^{\mu}q^{\nu}$ factor).
One can extend this to non-zero momentum transfer
\begin{equation}
  e(q^2) = (1 + \Pi_2(q^2))^{1/2}e_0\,,
\end{equation}
so we have an effective charge for non-zero photon momenta.

To see how our coupling $\alpha \sim e^2$ ``evolves'' as we go to higher momenta, we consider
\begin{equation}
  \frac{\alpha(q^2)}{\alpha(0)} = \frac{1 + \Pi_2(q^2)}{1 + \Pi_2(0)}\,.
\end{equation}
Focusing on large space-like photon momenta,
we find qualitatively that the coupling grows (although only weakly).
In fact there is a pole, called the Landau pole, at which the coupling diverges,
but this is of limited physical significance
because the perturbative approach we used to calculate the coupling
breaks down as soon as the coupling is no longer small.

The physical interpretation of this effect is that
at low-momenta one probes the charge from long distances.
The bare charge passively produces electron-positron pairs, polarizing the vacuum,
and these form dipoles to screen the bare charge.
So at high momenta, the photon ``sees'' more of the bare charge.
This picture works well for QED, but other theories, such as QCD, also have running couplings,
and an image of charge screening is not as obviously applicable there.


\section{The Callan-Symanzik equation}

The starting point of the scaling analysis is fixing a renormalization scale $M$.
Setting reasonable renormalization conditions consistent with this scale allows one to compute renormalized Green's functions $G^{(n)}$.
The underlying theory gives bare Green's functions, which are related to renormalized Green's functions by renormalization constants.
So we could consider a different renormalization scale which gives different renormalized Green's functions,
which are related to the same bare Green's functions.

One way to do this is by shifting our scale $M$ and shifting our field strengths and couplings (or masses)
such that the bare Green's functions are still the same.
Then our renormalized Green's function is rescaled according to the necessary field rescaling.
Thus, we can think of $G^{(n)}$ as being a function of the scale $M$ and the coupling(s) $\lambda$,
so the change in $G$ is given by
\begin{equation}
  n \delta \eta G^{(n)} = \frac{\partial}{\partial M}G^{(n)} \delta M + \frac{\partial}{\partial \lambda} G^{(n)} \delta \lambda \,.
\end{equation}

This gives the Callan-Symanzik equation
\begin{equation}
  \left[M \frac{\partial}{\partial M} + \beta(\lambda) \frac{\partial}{\partial \lambda} + n \gamma(\lambda)\right]G^{(n)}(\{x_i\}; M, \lambda) = 0\,.
\end{equation}
$\beta$ and $\gamma$ are universal functions, the same for every $n$, independent of $M$,
related to how the field strength and coupling constant must change if the renormalization scale is changed.
Note: $\gamma$ is called the anomalous dimension, because it sets how far the scaling of the Green's function is
from canonical scaling.

\section{The $\beta$ function}

The beta function is given by
\begin{equation}
  \beta = M \frac{\partial \lambda}{\partial M}\,,
\end{equation}
or alternatively
\begin{equation}
  \beta(\lambda(t)) = t \frac{\partial \lambda(t)}{\partial t}\,.
\end{equation}
It has a trivial fixed point at 0 coupling
and its value for small couplings says a lot about the momentum dependence of the theory.
At fixed points, the coupling becomes scale independent.

\chapter{The path integral}

\section{Key ideas}

\begin{itemize}
  \item So far, our approach to quantization and evaluation of observables has been
    to quantize the theory by demanding canonical (anti-) commutation relations
    and compute observables perturbatively by evaluating the relevant diagrams by hand
  \item An alternative approach to quantization occurs via the path integral,
    also applicable to classical systems
  \item This formalism contains all correlation functions in one object,
    allowing the calculation of observables in principle up to complete order
  \item The central object is the generating functional,
    which couples classical sources to fields of the theory
  \item Some difficulties arise when dealing with the quantization of fermions,
    which require the introduction of Grassmann fields to ensure anti-commuting behavior
  \item The path integral also allows for the easy identification of the propagator,
    but for massless gauge fields this is spoiled by the fact that the inverse propagator is singular
  \item Fixing a gauge makes the inverse propagator invertible (recall previously we worked in the Feynman gauge)
  \item More generally, the Faddeev-Popov method allows for the fixing of a gauge
    for the functional integral over gauge fields,
    which results in the modification of the inverse propagator by a gauge-fixing term
  \item Specific choices for this gauge-fixing-term parameter $\xi$ correspond to different gauges,
    and we recover the Feynman gauge result used previously
\end{itemize}

\section{Application to scalar field theory}

\subsection{Classical case}

\subsection{Path integral quantization}

The derivation of the path integral proceeds in analogy to the derivation for the case
with a classical point particle.
Time is discretized to handle the propagation as before.
Additionally, space is discretized as each point in space of the field is treated as an independent degree of freedom.
The identity at each time point is inserted to reduce the canonical momentum operators and the field operators
to momentum fields and regular fields.
A subtle point to note is that in this case the canonical momentum is not simply the time derivative of the field,
thus the Gaussian integral to recover the Lagrangian requires completing the square.
This introduces an overall normalization, which ultimately will cancel.

Thus the path integral is given by
\begin{equation}
  \langle \Omega \vert T[\Phi_H(x_1) \Phi_H(x_2)] \vert \Omega \rangle
  = \frac{\int \mathcal{D}\Phi \Phi(x_1) \Phi(x_2) \exp(i S[\Phi])}{\int \mathcal{D}\Phi \exp(i S[\Phi])}\,.
\end{equation}

For something like the $\Phi^4$ theory,
one can start by considering the free Klein-Gordon Lagrangian.
In this case, one directly recovers the typical disconnected results for $n$-point functions
in terms of the propagator $D_F(x-y)$.
A perturbative expansion of the interaction term in the path integral shows
that one recovers at NLO the disconnected diagrams with 1-vertex vacuum bubbles
and then appropriate 1-vertex connected diagrams.
The denominator in the path integral above cancels all of the disconnected diagrams and vacuum bubbles
as was previously argued.
Thus, we have completely recovered the previous results for correlation functions in the path integral.

A couple comments:
\begin{itemize}
  \item This object essentially unites all correlation functions.
    The approach is analogous for other correlation functions.
  \item Exact evaluation of the path integral corresponds to a calculation of the correlation function
    amplitude to all orders.
  \item This approach works also for non-perturbative problems.
  \item By modifying the action, one can impose additional constraints on the system,
    for example the presence of an external field.
    One simply needs to specify how it couples to the relevant fields.
  \item The path integral only contains classical fields, not operators.
    All quantum fluctuations are captured in the functional integral,
    with configurations away from the classical path weighted differently by the action.
\end{itemize}

\subsection{Generating functional}

To simplfy the path integral even further, one can define a generating functional.
This uses some simple functional differential calculus tools to contain all correlation functions in one object.
As a reminder, some useful properties of functional derivatives are:
\begin{align}
  \frac{\delta f(y)}{\delta f(x)} & = \delta^4(x - y)\,, \\
  \frac{\delta}{\delta f(x)} \int d^4y f(y) g(y) &= g(x)\,, \\
  \frac{\delta}{\delta f(x)} \exp(i \int d^4y f(y) g(y)) &= i g(x) \exp(i \int d^4y f(y) g(y))\,, \\
  \frac{\delta}{\delta f(x)} \int d^4y (\partial_{\mu} f(y)) v^{\mu}(y) & = - \partial_{\mu} v^{\mu}(x)\,,
\end{align}
where we used integration by parts with vanishing surface terms in the last equality.

For our scalar case, the generating functional simply extends the Lagrangian
by a local term coupling our field $\Phi$ to a classical source $J$:
\begin{equation}
  Z[J] = \int \mathcal{D}\Phi \exp(i \int d^4x (\mathcal{L}[\Phi] + J(x) \Phi(x)))\,.
\end{equation}
We immediately recover the denominator of our path integral by evaluating $Z[J=0]$.
Additionally, we get the numerator for any $n$-point function
simply by taking functional derivatives with respect to $J$ at appropriate points
and setting $J=0$ at the end as well.

\section{Application to QED}

Adapting the path integral formalism to QED comes with some complications.
The first is that fermions, by the spin-statistics theorem, obey anti-commutation relations,
which means in the path integral we need anti-commuting scalar fields.
The second is that we now have a gauge field
and a Lagrangian that at least initially isn't fixed to a specific gauge.
This means we will have some difficulty getting the propagator our of the path integral,
and eventually we will enforce a criterion that fixes a specific gauge.

\subsection{Quantizing fermions}

Grassmann numbers are anti-commuting scalars.
Let $\theta$ and $\eta$ be Grassmann numbers.
Then,
\begin{align}
  \theta \eta & = - \eta \theta\,, \\
  \theta^2 & = 0\,, \\
  \frac{d}{d\theta}\theta\eta & = \eta\,,\\
  \frac{d}{d\eta}\theta\eta & = -\theta\,.\\
\end{align}
One consequence of these properties is that the Taylor series of a function of a Grassmann variable
only has a constant and a linear term.
Additionally, the definition of integration changes such that integrating is the same as differentiating.

Eventually, we use Grassmann fields
\begin{equation}
  \Psi(x) = \sum_i \theta_i \phi_i(x)\,,
\end{equation}
with basis functions $\phi_i(x)$.

\subsection{Quantizing the electromagnetic gauge field}

To simplify the discussion, we will focus on a pure gauge Lagrangian taken from QED,
\begin{equation}
  \mathcal{L}_0 = -\frac{1}{4} F_{\mu\nu}F^{\mu\nu}\,.
\end{equation}
By integrating by parts, we can get the action to a form,
\begin{equation}
  S[A] = \frac{1}{2}\int d^4 x (A_{\mu} (\partial_{\sigma} \partial^{\sigma} g^{\mu\nu} - \partial_{\mu} \partial^{\nu}) A_{\nu})\,.
\end{equation}
The gauge field propagator is the Green's function of the operator between the fields above.
In momentum space, this amounts to solving
\begin{equation}
  (-k^2 g_{\mu\nu} + k_{\mu} k_{\nu}) D_{F}^{\nu\rho}(k) = i \delta_{\mu}^{\rho}\,,
\end{equation}
which is not possible since $-k^2 g_{\mu\nu} + k_{\mu} k_{\nu}$ is singular.

The problem lies in the fact that our current functional integral integrates over all gauge field configurations,
even those that are gauge equivalent.
The Faddeev-Popov trick is to introduce a functional integral over gauge fields that gives 1,
inserting this into the partition function.
This is
\begin{equation}
  1 = \det\left(\frac{\delta G(A^{\alpha})}{\delta \alpha}\right) \int \mathcal{D}\alpha \delta [G(A^{\alpha})]\,,
\end{equation}
where the functional determinant is independent of $A$ and $\alpha$, so we can pull it out and ignore it.

We get our partition function
\begin{equation}
  Z = \int \mathcal{D}\alpha \int \mathcal{D}A \exp(i S[A]) \delta [G(A^{\alpha})]\,,
\end{equation}
ignoring the determinant.
By gauge invariance, the action is unchanged ($S[A] = S[A^{\alpha}]$),
and, because gauge transformed field is just shifted by an $A$-independent term,
the integration metric is also unchanged.
This gives
\begin{equation}
  Z = \int \mathcal{D}\alpha \int \mathcal{D}A \exp(i S[A]) \delta [G(A)]\,.
\end{equation}
The $\alpha$ integral is now also only an overall normalization, since the integrand is independent of $\alpha$.

Now, what remains is to:
\begin{enumerate}
  \item Specify a gauge fixing function, for example $G(A) = \partial^{\mu} A_{\mu}(x) - \omega(x)$.
  \item Integrate over $\omega$ with a Gaussian weight and a normalization such that the integral is an identity.
    The normalization will cancel when we consider a full path integral.
  \item Evaluate $\omega$ integral, bringing a gauge fixing term into the action.
\end{enumerate}
For the full path integral,
this only works if the operators in the correlation function of interest are gauge invariant.

The resulting inverse propagator is now $-k^2 g_{\mu\nu} + (1 - \frac{1}{\xi})k_{\mu} k_{\nu}$,
where $\xi$ is the width of our Gaussian weighting function.
This lifts the singularity,
giving the momentum-space propagator
\begin{equation}
  D_{F}^{\mu\nu}(k) = \frac{-i}{k^2 + i \epsilon}\left(g^{\mu\nu} - (1 - \xi) \frac{k^{\mu}k^{\nu}}{k^2}\right)\,.
\end{equation}

\section{Symmetries and Ward identities}

\section{Extending the generating functional concept}

\subsection{The Schwinger functional}

\subsection{The effective action}

\subsection{Useful properties}

\chapter{Spontaneous symmetry breaking}

\section{Key ideas}

\begin{itemize}
  \item Spontaneous symmetry breaking (SSB) is when a theory has a global symmetry
    that is not realized in the ground state
  \item A simple classical example is the sombrero potential,
    which has a 2-dimensional rotational symmetry that the degenerate ground states break
  \item This has the consequence that for the broken continuous symmetries
    there are 0-energy excitations from the ground state into another ground state,
    which in a quantum field theory view would correspond to massless particles
  \item One such theory is the linear $\sigma$ model,
    in which the theory has an $\text{O}(N)$ symmetry,
    but the ground state only has an $\text{O}(N-1)$ symmetry,
    giving rise to $N-1$ ``Goldstone bosons''
  \item Quantizing the theory and renormalizing it gives the result that
    these bosons are still massless and the Goldstone theorem remains intact
\end{itemize}

\section{Classical argument}

The Lagrangian for the linear $\sigma$ model in Eq.~\ref{eq:linear_sigma_model} has a couple of interesting/unusual properties:
\begin{enumerate}
  \item the ``mass term'' has the wrong sign,
  \item and the Lagrangian is invariant under $N$-dimensional rotations ($\text{O}(N)$).
\end{enumerate}
Using $\mathcal{H} = \pi^2 - \mathcal{L}$, we can identify the classical potential with
\begin{equation}
  V(\Phi^2) = - \frac{1}{2} \mu^2 \Phi^2 + \frac{\lambda}{4} \Phi^4\,.
\end{equation}
The classical ground state is a constant field $\lvert\Phi\rvert = \nu$ that minimizes the potential,
which happens for
\begin{equation}
  \nu^2 = \frac{\mu^2}{\lambda}\,.
\end{equation}
It is easy to see that this ground state is not unique and not invariant under an $N$-dimensional rotation.

The consequences of this are made clear when we shift our fields to work around one of our classical minima:
\begin{align}
  \Phi^i(x) & = \pi^i(x)\,,  &i \in \left\{1,\ldots,N-1\right\}\\
  \Phi^N(x) &=  \nu + \sigma(x)\,. &
\end{align}
Re-expressing the Lagrangian in terms of these shifted fields shows that only the $\sigma$ mode has a ``mass'' term,
and the $N-1$ $\pi$ fields correspond to 0-energy excitations of the ground state.

This is one realization of the Goldstone theorem:
For every spontaneously broken continuous symmetry there exists a 0-energy excited state.
In a quantized theory, we would identify these with massless particles, the so-called ``Goldstone bosons.''
The number of unique 0-energy excitations can be determined via the reduction of generators
going from the full Lagrangian symmetry group to the symmetry group of the ground state.
In this example, the group $\text{O}(N)$ has $\frac{1}{2}(N-1)N$ generators,
so this would mean we get $N-1$ 0-energy excitations, which is what we just observed.

The classical proof of this can be obtained schematically as follows:
\begin{enumerate}
  \item Expanding the potential around a minimum $\Phi_0^i$, we identify the 2-derivative matrix as a mass matrix $m^2_{ab}$.
  \item If we perform an infinitesimal symmetry transformation of the fields
    (which leaves the potential unchanged)
    and evaluate at the minimum,
    we find ultimately that the product of the mass matrix and the vector giving the shift of the fields must be 0.
  \item In the case that the rotation is around an axis not equal to $\Phi_0^i$, this is a non-trivial condition,
    which means that the infinitesimal shift vector is an eigenvector of the mass matrix with eigenvalue 0.
\end{enumerate}



\section{Quantization and renormalization}

\chapter{The gauge principle}

\section{Key ideas}

\begin{itemize}
  \item Previously, we arrived at the QED Lagrangian by explicitly coupling a field to our fermions,
    adding a kinetic term for the new field,
    and demanding specific transformation behavior under gauge transformations
  \item An alternative approach is to take the free Lagrangian with a global symmetry,
    $\text{U}(1)$ in the case of QED,
    promote this to a local symmetry and require that the Lagrangian is invariant under it
  \item This approach, called ``gauging,'' naturally generates the gauge fields and ultimately gives us the same result,
    but it gives a way to systematically construct the QED Lagrangian as well as more complicated theories
\end{itemize}

\section{QED from gauge invariance}

Starting from the free Dirac Lagrangian in Eq.~\ref{eq:dirac_free},
we want to get the QED Lagrangian
that is invariant under a local $\text{U}(1)$ gauge transformation of the fields
\begin{equation}
  \Psi(x) \rightarrow e^{i \alpha(x)}\Psi(x).
\end{equation}

The approach taken is as follows:
\begin{enumerate}
  \item The mass term $\bar{\Psi}m\Psi$ is already invariant.
    The derivative as written in an $\epsilon$ limit is not,
    since any two spacetime points may transform differently.
  \item Thus we define an object $U(y, x)$
    which, when transformed,
    undoes the transformation at one spacetime point $x$
    and applies the transformation at another spacetime point $y$.
    Applying this to our $\epsilon$-expanded derivative, we define a covariant derivative,
    which transforms consistently.
  \item Expanding everything in powers of $\epsilon$ gives a normal derivative term
    and a non-derivative term proportional to the derivative of $U$ at $x$.
    This can be defined into a new field whose transformation behavior is given by the transformation behavior of $U$.
  \item This gives us covariant derivative of the field, which transforms just like the field itself.
    Now one can combine fields and derivatives of fields to produce gauge invariant building blocks.
  \item The final step involves obtaining the kinetic term for the new field.
    One possible way to arrive at this is noting that $[D_{\mu}, D_{\nu}]$ is invariant under gauge transformations.
    This term can be simplified into $F_{\mu\nu}=\partial_{\mu}A_{\nu} - \partial_{\nu}A_{\mu}$.
    Lorentz invariance requires that this only appears as $(F_{\mu\nu})^2$ or contracted with a Levi-Civita tensor,
    and the latter term can be shown not to contribute to the action and can be left out as a result.
\end{enumerate}

\section{The Higgs mechanism}

\chapter{Non-abelian gauge theories}

\section{Key ideas}

\begin{itemize}
  \item When gauging QED, our final expressions were simplified by the fact that the generators of $\text{U}(1)$ commute
  \item In non-abelian gauge theories, the symmetry group is non-abelian,
    meaning the group elements (and thus the generators of the group) do not in general commute
  \item This leads to some additional complexities when gauging these theories
\end{itemize}

\section{Yang-Mills theory}

\section{QCD}

The ``free'' QCD Lagrangian is given by
\begin{equation}\label{eq:qcd_free}
  \mathcal{L} = \sum_{f} \bar{\Psi}_f (i \slashed{\partial} - m_f) \Psi_f\,,
\end{equation}
where our $\Psi$ fields are triplets of Dirac spinors, corresponding to color degrees of freedom.

\subsection{Symmetries of QCD}

The Lagrangian above has a global $\text{SU}(3)$ symmetry,
which we will promote to a local gauge symmetry.
The $\text{SU}(3)$ generators are $T^a=\lambda^a / 2$, which means we now demand invariance under
\begin{equation}
  \Psi_f(x) \rightarrow \exp(i\theta^a(x)T^a)\Psi_f(x).
\end{equation}
This gives the covariant derivative
\begin{equation}
  D_{\mu} = \partial_{\mu} - i g A_{\mu}^a(x) T^a,
\end{equation}
where we have generated the 8 gluon gauge fields.
The transformation behavior of these fields is a bit more complicated,
\begin{equation}
  A_{\mu}^a(x) \rightarrow A_{\mu}^a(x) + \frac{1}{g} \partial_{\mu} \theta^a(x) + f^{abc}A_{\mu}^b(x)\theta^c(x),
\end{equation}
directly a result of the fact that the generators do not commute.

Once again, as in QED, the field strength tensor
\begin{equation}
  G_{\mu\nu}^a = \partial_{\mu} A_{\nu}^a - \partial_{\nu} A_{\mu}^a  + g f^{abc} A_{\mu}^b A_{\nu}^c
\end{equation}
is invariant under local transformations.
Thus the Lorentz invariant term we can include in our gauged QCD Lagrangian is $G_{\mu\nu}^a G^{\mu\nu a}$.
The first 2 terms in the field strength tensor gives us a kinetic term for the gluons.
However, the final term in $G_{\mu\nu}$ generates 3-gluon ($\mathcal{O}(g)$) and 4-gluon terms ($\mathcal{O}(g^2)$), due to its quadratic appearance.

There is also a term that goes like
\begin{equation}
  \varepsilon^{\mu\nu\rho\sigma}G_{\mu\nu}^a G_{\rho\sigma}^a \,,
\end{equation}
known as the $\theta$ term.
This term exists in principle,
but experimental evidence suggests it is very small.


\subsection{Quantization}


\begin{appendices}
  \chapter{Some useful equations to remember}

  \section{Lagrangians}

  Free Klein-Gordon theory:
  \begin{equation}\label{eq:kg_free}
    \mathcal{L} = \frac{1}{2}(\partial^{\mu} \Phi)(\partial_{\mu} \Phi) - \frac{m^2}{2} \Phi^2\,.
  \end{equation}

  $\Phi^4$ theory:
  \begin{equation}\label{eq:phi4}
    \mathcal{L} = \frac{1}{2}(\partial^{\mu} \Phi)(\partial_{\mu} \Phi) - \frac{m^2}{2} \Phi^2 - \frac{\lambda}{4!} \Phi^4\,.
  \end{equation}

  Linear $\sigma$ model:
  \begin{equation}\label{eq:linear_sigma_model}
    \mathcal{L} = \frac{1}{2}(\partial^{\mu} \Phi^i)(\partial_{\mu} \Phi^i) + \frac{\mu^2}{2} \Phi^2 - \frac{\lambda}{4} \Phi^4\,,
  \end{equation}
  with $\Phi^i$ an $N$-dimensional vector of scalar fields.

  Free Dirac theory:
  \begin{equation}\label{eq:dirac_free}
    \mathcal{L} = \bar{\Psi}(i\slashed{\partial} - m)\Psi\,.
  \end{equation}

  QED:
  \begin{equation}\label{eq:qed}
    \mathcal{L} = \bar{\Psi}(i\slashed{D} - m)\Psi - \frac{1}{4}(F_{\mu\nu})^2 \,,
  \end{equation}
  with
  \begin{equation}\label{eq:qed_cov_deriv}
    D_{\mu} = \partial_{\mu} + i e A_{\mu}\,.
  \end{equation}
  The gauge field transforms like
  \begin{equation}
    A_{\mu}(x) \rightarrow A_{\mu}(x) + \frac{1}{e} (\partial_{\mu} \alpha(x))\,.
  \end{equation}

  ``Free'' QCD:
  \begin{equation}
    \mathcal{L} = \sum_f \bar{\Psi}_f (i\slashed{\partial} - m_f) \Psi_f \,.
  \end{equation}

  QCD:
  \begin{equation}
    \mathcal{L} = \sum_f \bar{\Psi}_f (i\slashed{D} - m_f) \Psi_f - \frac{1}{4} G_{\mu\nu}^a G^{\mu\nu a}
    + \theta \varepsilon^{\mu\nu\rho\sigma} G_{\mu\nu}^a G_{\rho\sigma}^a \,,
  \end{equation}
  with
  \begin{equation}
    D_{\mu} = \partial_{\mu} - i g A_{\mu}^a (x) T^a\,,
  \end{equation}
  and
  \begin{equation}
    G_{\mu\nu}^a = \partial_{\mu} A_{\nu}^a - \partial_{\nu} A_{\mu}^a  + g f^{abc} A_{\mu}^b A_{\nu}^c \,.
  \end{equation}
  The gluon gauge field transforms like
  \begin{equation}
    A_{\mu}^a(x) \rightarrow A_{\mu}^a(x) + \frac{1}{g} (\partial_{\mu} \theta^a(x)) + f^{abc} A^b_{\mu}(x) \theta^c(x)\,.
  \end{equation}
  It is also possible to add electromagnetic interactions to QCD by extending the covariant derivative
  by the same term as for QED and including the EM field strength tensor term.

  \section{Feynman rules}

  \subsection{$\Phi^4$}

  \begin{align}
    \text{propagator} &= \frac{i}{p^2 - m^2 + i \varepsilon}\,, \\
    \text{vertex} &= -i \lambda\,.
  \end{align}

  \subsection{QED (with counter terms)}
  \begin{align}
    \text{fermion propagator} &= \frac{i}{\slashed{p} - m + i \varepsilon}\,,
                              & \text{counter term} &= i (\slashed{p}\delta_2 - \delta_m)\,, \\
    \text{photon propagator} &= \frac{-i g^{\mu\nu}}{q^2 + i \varepsilon}\,,
                              & \text{counter term} &= - i (g^{\mu\nu}q^2 - q^{\mu}q^{\nu}) \delta_3\,, \\
    \text{vertex} &= -i e \gamma^{\mu}\,, & \text{counter term} &=-i e \gamma^{\mu} \delta_1\,.
  \end{align}

  \subsection{Linear $\sigma$ model}

  \subsection{QCD}

  \begin{align}
    \text{gluon-fermion vertex} &= i g \gamma^{\mu} T^a\,, \\
    \text{3 gluon vertex} &= g f^{abc}(\ldots)\,, \\
    \text{4 gluon vertex} &= i g^2 (\text{terms like} \sim f^{abe}f^{cde} \ldots)\,.
  \end{align}
  The fermion and gauge field propagators are roughly the same as in QED.
  Because the Faddeev-Popov method for QCD does not easily get rid of the determinant in the inserted identity
  there are also Faddeev-Popov ghost fields with some Feynman rules.
  These only appear as propagators (not as external lines).




  \chapter{The spectral representation}

\end{appendices}

\backmatter{}

\bibliographystyle{alpha} % use your favorite BIBTeX style
\nocite{*} % To display all refs, even uncited refs (useful when editting)
\bibliography{notesbib}

\end{document}
